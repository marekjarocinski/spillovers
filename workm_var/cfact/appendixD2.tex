\begin{filecontents}{spillovers.bib}
@TechReport{McKay_Wolf_2022,
  author={Alisdair McKay and Christian Wolf},
  title={What Can Time-Series Regressions Tell Us About Policy Counterfactuals?},
  year=2022,
  institution={MIT},
  type={manuscript},
  url={https://www.christiankwolf.com/research}
}
@Article{Leeper_Zha_2003,
  author={Leeper, Eric M. and Zha, Tao},
  title={{Modest policy interventions}},
  journal={Journal of Monetary Economics},
  year=2003,
  volume={50},
  number={8},
  pages={1673-1700},
  month={November},
  keywords={},
  doi={10.1016/j.jmoneco.2003.01.002}
}



\end{filecontents}

\documentclass[a4paper,12pt]{article}
\usepackage{amsfonts}
\usepackage{amsmath}
\usepackage{graphicx}
\usepackage{amssymb}
\usepackage{array}
\usepackage[round]{natbib}
\usepackage{booktabs}
\usepackage{floatpag}
\usepackage[hmargin=2.5cm,vmargin=3cm]{geometry}
\usepackage{setspace}
\usepackage{chngcntr}
\usepackage{verbatim}
\usepackage{multirow}
\usepackage{url}
\usepackage{doi}
\usepackage{hyperref}
\usepackage{xcolor}
\hypersetup{
    colorlinks,
    linkcolor={blue!50!black},
    citecolor={blue!50!black},
    urlcolor={blue!80!black}
}

\usepackage[gen]{eurosym}
\floatpagestyle{plain}
\newcolumntype{H}{>{\setbox0=\hbox\bgroup}c<{\egroup}@{}}
\renewcommand{\floatpagefraction}{0.95}
\setstretch{1.5}
\newcommand{\pathTables}{../workm_lp/}
\newcommand{\pathFigures}{}
\newcommand{\pathA}{}
\newcommand{\pathB}{}
\newcommand{\pathC}{}
\begin{document}
\pagenumbering{gobble}

\author{Marek Jaroci\'nski}
\date{\today}
\title{Counterfactual: ECB monetary policy spillover shutting down the Fed response}
\maketitle




This section reports the counterfactual effect of an ECB monetary policy shock on the US
in the scenario in which the Fed monetary policy does not respond to offset the shock.
This exercise, taken at face value, would suggest that in the absence of the Fed offsetting policy,
the ECB monetary policy shocks would spill over across the Atlantic as strongly as the Fed monetary policy shocks do.

I take the Wu-Xia Shadow Rate as the summary of the Fed monetary policy stance.
I run a similar VAR for the US as in the main paper, except that this time I include
the ECB monetary policy shock $i^{MP,ECB}$ along with the Fed monetary policy shock $i^{MP,Fed}$
(for simplicity I use the median shocks in this exercise).
Then, for each draw from the posterior distribution of the impulse responses I construct a sequence of Fed monetary policy shocks that exactly offsets the response of the Shadow Rate to the ECB monetary policy shock.


\begin{figure}[!htbp]
\caption{The effect of ECB shocks on the US variables: counterfactual.}\label{fig: var counterfactual1}
\newcommand{\myfig}[1]{\includegraphics[width=0.33\textwidth]{#1}}
\renewcommand{\pathA}{../cfact/us-cfact0}
\renewcommand{\pathB}{../cfact/us-cfact1}
\renewcommand{\pathC}{../cfact/us-spol}
\newcommand{\myrow}[1]{\myfig{\pathA-#1} & \myfig{\pathB-#1} & \myfig{\pathC-#1}}
\begin{center}
\begin{tabular}{ccc}
$i^{MP,ECB}$ Actual  & $i^{MP,ECB}$ {Counterfactual} & Memo: $i^{MP,Fed}$ \\
%\myrow{MP_median_ecb} \\
%\myrow{MP_median_fed} \\
\myrow{us_wuxia} \\
\myrow{sp500_a} \\
\myrow{bofaml_us_hyld_oas_a} \\
\myrow{eurusd_a} \\
\myrow{broadexea_usd_a} \\
\myrow{us_rgdp}\\
\myrow{us_gdpdef}\\
\end{tabular}
\end{center}
\footnotesize Note: The red solid-dotted lines represent the point-wise posterior medians of the impulse responses to the central bank information shock. The red areas show the pointwise 16-84 percentile bands. 
The blue solid lines and blue areas show the same objects for the monetary policy shock. 
The figure is based on 10,000 draws from the Gibbs sampler.
\end{figure}

Figure \ref{fig: var counterfactual1} reports this exercise.
The first column shows, as the reference point, the spillover of a one standard deviation ECB monetary policy shock, familiar from Figure 4 in the paper.
The Shadow Rate declines and the responses of the US variables are muted.
The euro appreciates against the dollar.
The second column shows the counterfactual spillover of the same ECB monetary policy shock.
Now the Shadow Rate does not move by construction.
We can see that stock prices decline, bond spreads increase. The euro appreciation
is this time muted. The dollar appreciates against the currencies other than the euro,
consistently with a risk-off scenario. Real GDP and its deflator decline.
The third column shows, for comparison, the effects of a one standard deviation Fed monetary policy shock.\footnote{These responses are very close but not identical to the blue responses in Figure D.3 in the Online Appendix.
Here I plot the responses to the Fed monetary policy shock obtained with the median decomposition at the FOMC announcement frequency and then the obtained shock is aggregated to the monthly frequency.
By contrast, in Figure D.3 in the Online Appendix I first aggregate the FOMC announcement \emph{surprises}
to the monthly frequency and then impose the sign restrictions in the VAR, accounting for the uncertainty about the rotations.
This approach produces slightly larger uncertainty bands. Otherwise the differences in the impulse responses obtained with these two approaches are very small.}
Comparing columns two and three we can see that in the absence of the Fed offsetting policy response,
the spillover of the ECB monetary policy shock to the US is quite similar to the domestic effect of a Fed monetary policy shock. Comparing with Figure 11 in the paper we can conclude that in this counterfactual scenario ECB monetary policy spills over across the Atlantic as well as the Fed monetary policy does.

This exercise is only illustrative and the stark finding should be taken cautiously for several reasons.
First, there are limits to which we can simulate changes in systematic policy using sequences of identified shocks, especially in the case of material changes in systematic policy \citep{Leeper_Zha_2003}.\footnote{\cite{McKay_Wolf_2022} show to overcome these limits by using \emph{policy news} shocks to construct the counterfactual, but identifying such shocks would be beyond the scope of this paper.} Second, the exercise strongly depends on the choice of the variable that summarizes monetary policy stance. 
Taking either the Krippner's Shadow Rate, the Fed Funds rate or the Treasury yield instead of the Wu-Xia Shadow Rate results in even larger error bands around the counterfactual. Third, even in the current specification the error bands in the counterfactual scenario are very large.

\bibliographystyle{econ}
\bibliography{spillovers}

\end{document}
